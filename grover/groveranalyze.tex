%% -*- coding:utf-8 -*- 
\subsection{Analysis of Grover's Algorithm}

The schematic form of Grover's algorithm is given in Alg.
\ref{alg:quantcomp:grover}.  
\begin{algorithm}
\caption{Grover's Algorithm}
\label{alg:quantcomp:grover}
\begin{algorithmic}
    \STATE $\left|\psi\right>_0 \Leftarrow \frac{1}{\sqrt{N}}\sum_x 
    \ket{x}$
    \STATE $t \Leftarrow 1$
    \REPEAT
    \STATE $\left|\psi\right>_t \Leftarrow \hat{U}_s\hat{U}_{x^{\ast}}
    \left|\psi\right>_{t-1}$
    \STATE $t \Leftarrow t + 1$
    \UNTIL{ ($t < \frac{\pi}{4}\sqrt{N}$)}
    \RETURN result of measuring the state $\left|\psi\right>_t$
\end{algorithmic}
\end{algorithm}

We will be interested in two questions: what is the algorithmic complexity
of Grover's algorithm and are there algorithms that can perform
the search task in an unstructured data set more efficiently than
Grover's algorithm.

The criterion of algorithm efficiency is the following fact: a good
algorithm should find the desired value with a minimum number of function
calls \eqref{eqQuantCompGroverF}.

Let us consider the very first iteration. The initial state
$\left|\psi\right>_0$ has the following form
\begin{equation}
\left|\psi\right>_0 =
\sum_x \alpha_x \ket{x} =  
\ket{s} = 
\frac{1}{\sqrt{N}}\sum_x \ket{x} = 
\frac{1}{\sqrt{N}}\sum_{x\ne x^{\ast}} \ket{x} +
\frac{1}{\sqrt{N}} \left|x^{\ast}\right>.
\nonumber
\end{equation}
Thus, the coefficient before the target element has the form
$\alpha_x^{\ast} = \frac{1}{\sqrt{N}}$. 

After applying the phase inversion operator
$U_{x^{\ast}}$ from \eqref{eqQuantCompGroverUxast} we get
\begin{equation}
\hat{U}_{x^{\ast}} \left|\psi\right>_0 =
\frac{1}{\sqrt{N}}\sum_{x\ne x^{\ast}} \ket{x} - 
\frac{1}{\sqrt{N}} \left|x^{\ast}\right> = \sum_x \beta_x \ket{x},
\nonumber
\end{equation}
where $\beta_{x^\ast} = - \frac{1}{\sqrt{N}}$ and $\beta_{x \ne x^\ast} =
\frac{1}{\sqrt{N}}$. 

After applying the inversion about the mean operator $\hat{U}_s$ 
from \eqref{eqQuantCompGroverUs} we get
\begin{eqnarray}
\hat{U}_G \left|\psi\right>_0 = 
\hat{U}_s \hat{U}_{x^{\ast}} \left|\psi\right>_0 = 
\hat{U}_s \sum_x \beta_x \ket{x} = 
\nonumber \\
= \sum_x \left(2 M - \beta_x\right) \ket{x} \approx 
\sum_{x\ne x^{\ast}} \left( 2 \frac{1}{\sqrt{N}} - \frac{1}{\sqrt{N}}
\right) \ket{x} + 
\nonumber \\
+ \left( 2 \frac{1}{\sqrt{N}} +
\frac{1}{\sqrt{N}} \right) \left|x^{\ast}\right> = 
\frac{1}{\sqrt{N}}\sum_{x\ne x^{\ast}} \ket{x} + 
\frac{3}{\sqrt{N}} \left|x^{\ast}\right>.
\label{eqQuantCompGroverAnalyzeRought}
\end{eqnarray}
In deriving \eqref{eqQuantCompGroverAnalyzeRought}, it was assumed that
\[
\mathcal{M} = \frac{\sum_x \alpha_x}{N} \approx
\frac{N}{N \sqrt{N}} = \frac{1}{\sqrt{N}}.
\]

Thus after the first iteration of Grover's algorithm, the amplitude
$\alpha_{x^{\ast}}$ increased by $\frac{2}{\sqrt{N}}$. If
we approximate this result for an arbitrary iteration, we can
conclude that a $50\%$ probability of finding $\left|x^{\ast}\right>$
will be achievable after the following number of iterations:
\[
\frac{1}{\sqrt{2}}/\frac{2}{\sqrt{N}} =
\frac{\sqrt{N}}{2 \sqrt{2}} = O\left(\sqrt{N}\right).
\]

More accurate calculations \cite{nielsen2000quantum} give for the number of iterations
$\frac{\pi}{4}\sqrt{N}$. 

We may ask about the optimality of Grover's algorithm: is there
a quantum algorithm that performs a search in an unstructured
data set faster than $O\left(\sqrt{N}\right)$ function calls
\eqref{eqQuantCompGroverF}. The article
\cite{bBennettGroverOptimal} shows that no such algorithm
exists.