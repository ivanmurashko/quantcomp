%% -*- coding:utf-8 -*-
\section{Shor's Algorithm and the Discrete Logarithm on Elliptic Curves} 
Consider the elliptic curve 
\[
E\left(\mathbb{F}\right) = \{
(x,y) \in \mathbb{F}_p \times \mathbb{F}_p, y^2 = x^3 +a x + b,
\}
\]
with a given base point $g \in E\left(\mathbb{F}_p\right)$ such that: 
\[
n g = 0.
\]
The task is to solve the following: for the given $q
\in E\left(\mathbb{F}_p\right)$ find an $x$ such that
\begin{equation}
x g = q \mod n
\label{eq:quantcomp:discretlogeliptic}
\end{equation}

Consider the following auxiliary function
\begin{equation}
f(x_1, x_2) = x_1 q + x_2 g = \left(x x_1 + x_2\right) g,
\label{eq:quantcomp:discretlogeliptic:f}
\end{equation}
where $q,g \in E\left(\mathbb{F}_p\right)$ and are taken from the conditions of our
problem \eqref{eq:quantcomp:discretlogeliptic}. This function
is analogous to \eqref{eq:quantcomp:discretlogfunc}
used in solving the discrete logarithm problem. Then
we measure this function. The result of this measurement
is some point $c \in E\left(\mathbb{F}_p\right)$. Moreover, from
\eqref{eq:quantcomp:discretlogeliptic:f} it follows that 
$c \in \left<g\right>$, i.e., $\exists x_0$ such that $c = x_0 g$. 

Thus, by analogy with \eqref{eq:quantcomp:shordiscretlog:fprime}, we
compose the following function 
\begin{equation}
\label{eq:quantcomp:shorelliptic:fprime}
f'\left(x_1, x_2\right) = 
\begin{cases}
1, x x_1 + x_2 \equiv x_0 \mod n \\
0, x x_1 + x_2 \not\equiv x_0 \mod n \\
\end{cases}
\end{equation}

The coordinates $(j_1,j_2)$ of the maximum of the Fourier image $\tilde{f'}$ provide
according to the formula
\eqref{eq:quantcomp:discretlog:periodfourier} some value
of the desired number $x$. In our case, almost always $n \ne M$
so we can only use an approximate estimate
\[
x \approx \frac{j_1}{j_2}.
\]

\begin{example}[Discrete Logarithm on an Elliptic Curve]
Consider the problem from example \ref{ex:add:discretmath:ecdh}
The curve and base point (see example \ref{ex:add:elliptic:basepoint})
are as follows
\[
(p,a,b,g,n,h) = (97, -7, 10, (96,93), 41, 2)
\]
Suppose we know Alice's public key
\[
A = (37, 35)
\]
and we want to find an $x \in \{0,1, \dots 40\}$ such that
$x g = A$, as follows from example \ref{ex:add:discretmath:ecdh}
the answer is $x = d_a  = 5$.
Our function under investigation is
\[
f\left(x_1,x_2\right) = x_1 A + x_2 g = x_1 (37,35) + x_2 (96,93)
\]

\input elliptic/figellipticdiscretlog1.tex

As the measurement result, we choose $c = g$, i.e., $x_0 = 1$.
The graph of the function $f'(x_1, x_2)$, corresponding to this measurement
is shown in \autoref{fig:quantcomp:dle1}. Note that
the function $f'(x_1, x_2)$ is periodic and if we take any two closely
located points such as $(8,2)$ and $(16,3)$ we can see that
the period in coordinate $x_1$ is $T_1=8$, and in coordinate $x_2$ it is $T_2
= 1$. Solving the equation $x = T_2 T_1^{-1} \mod n$ we get $x = 8^{-1}
 \equiv 5 \mod 41$, which corresponds to the desired solution.

\input elliptic/figellipticdiscretlog2.tex

The Fourier image of the function $f'(x_1, x_2)$ is shown in
\autoref{fig:quantcomp:dle2}. From it, we can find the desired $x =
5$. 

\end{example}
