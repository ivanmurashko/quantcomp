\section{Description}
The course is dedicated to modern algorithms of classical encryption and new methods of breaking these algorithms through quantum computations.

The course consists of 10 lectures, each lasting 1 hour.

Lecture 1 contains a brief description of the course and an introduction to quantum mechanics.

Lecture 2 continues the introduction to quantum mechanics.

Lecture 3 contains a description of the basic principles of quantum computations.

Lecture 4 is dedicated to the description of the classical RSA algorithm and its connection to the problem of finding the function’s period.

Lecture 5 is dedicated to the discrete (classical) Fourier transform and how it can be used to find the period of functions. An implementation of the discrete Fourier transform using quantum elements is proposed.

Lecture 6 contains a description of Shor's algorithm for cracking RSA.

Lecture 7 describes symmetric encryption algorithms and Grover’s algorithm.

Lecture 8 is dedicated to classical encryption algorithms based on the complexity of the discrete logarithm problem. Elliptic curve cryptography algorithms are examined in detail.

Lecture 9 describes a modification of Shor's algorithm for solving the discrete logarithm problem.

Lecture 10 summarizes and describes the replacement of classical cryptography with quantum cryptography. Bell's experiment is considered, and how quantum probability theory differs from classical (Kolmogorov) probability. A quantum cryptography scheme based on Bell's experiment is proposed.

Throughout the lectures, necessary mathematical explanations will be provided, such as:
- Linear algebra and matrix operations: matrix multiplication, linear operators, eigenvalues and eigenfunctions of linear operators
- Discrete mathematics: Fermat's little theorem, Euclidean algorithm, etc.
- Classical probability theory: events, random variables, expected value of a random variable
