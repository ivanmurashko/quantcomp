\section{Discrete Logarithm}
\label{AddDiscretLog}

\begin{definition}[Discrete Logarithm]
Consider some abelian multiplicative group $G$ and the equation 
\begin{equation}
g^x = a
\label{eq:add:dm:discretlog}
\end{equation}
The solution to this equation, i.e., a non-negative integer $x$
satisfying the equation \eqref{eq:add:dm:discretlog}, is called
a discrete logarithm.
\end{definition}

Equation \eqref{eq:add:dm:discretlog} does not generally have a solution for any values of $a$. However, if $g$ is a generating element of $G$, i.e., $G=<g>$, then a solution always exists and is unique. When discussing discrete logarithms, we will assume that $g$ is chosen such that $G=<g>$.

In applied cryptography, we often deal with a special kind of
discrete logarithm in the ring of residues modulo $p$:
\begin{definition}[Discrete Logarithm in the Ring of Residues Modulo $p$]
The discrete logarithm $ind_g\left(a\right) \mod{p}$
\footnote{From the word {\bf ind}ex - an alternative name for the discrete logarithm}
is defined as the
smallest number $x$ that satisfies the following equation
(if such a number exists): 
\begin{equation}
g^x \equiv a \mod{p}
\end{equation}
\end{definition}

\begin{example}
\emph{($ind_3{14} \mod{17}$)}
Let's solve the problem by method of enumeration \cite{bWikiDiscretLog}:
\begin{eqnarray}
3^1 \equiv 3 \mod{17},\: 
3^2 \equiv 9 \mod{17},\: 
3^3 \equiv 10 \mod{17},\:
3^4 \equiv 13 \mod{17}, 
\nonumber \\
3^5 \equiv 5 \mod{17},\: 
3^6 \equiv 15 \mod{17},\: 
3^7 \equiv 11 \mod{17},\: 
3^8 \equiv 16 \mod{17}, 
\nonumber \\
3^9 \equiv 14 \mod{17},\: 
3^{10} \equiv 8 \mod{17},\: 
3^{11} \equiv 7 \mod{17},\: 
3^{12} \equiv 4 \mod{17}, 
\nonumber \\
3^{13} \equiv 12 \mod{17},\: 
3^{14} \equiv 2 \mod{17},\:
3^{15} \equiv 6 \mod{17},\: 
3^{16} \equiv 1 \mod{17},
\nonumber
\end{eqnarray}
Thus, it can be seen that $ind_3{14} \mod{17} = 9$, 
since $3^9 \equiv 14 \mod{17}$. 
\label{ex:dm:discretlog}
\end{example}

The problem of finding the discrete logarithm is a complex
task. The fastest known algorithm
\cite{bGordon93discretelogarithms} solves it in time on the order of 
\(
O\left(c \cdot
exp\left(\log{p}^{\frac{1}{3}}\log{\log{p}}^{\frac{2}{3}}
\right)\right)
\), where $c$ is some constant,
which is why algorithms using
discrete logarithms are widely applied in cryptography.