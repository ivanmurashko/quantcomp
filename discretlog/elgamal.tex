%% -*- coding:utf-8 -*-
\section{ElGamal Scheme}
\label{sec:add:dm:elgamal}
One of the variations of the Diffie-Hellman protocol is the ElGamal scheme. It is important to distinguish between the ElGamal encryption algorithm and the ElGamal digital signature algorithm. The ElGamal digital signature forms the basis of the digital signature standards of the USA (DSA) and Russia (GOST R 34.10-94).

Below we will consider the algorithm in encryption mode.

\subsection{Key Generation}

\begin{itemize}
\item A prime number $p$ is generated.
\item An integer $g$ is chosen.
\item A random integer $x: 1 < x < p$ is chosen.
\item $y = g^x \mod p$ is computed.
\end{itemize}

The public key is the triplet $p, g, y$. The private key is the number $x$.

\begin{example}[Key Generation (Elgamal)]
Let’s choose $p = 21, g = 10, x = 3$. $y = 10^3 \mod 21 = 13$.
\label{ex:add:dm:elgamal_gen}
\end{example}

\subsection{Encryption}
The message to be encrypted $M$ must satisfy the condition $0 < M < p$.
\begin{itemize}
\item A session key is chosen - a random number $k: 1 < k < p - 1$.
\item $a = g^k \mod p$ is computed.
\item $b = y^k M \mod p$ is computed.
\end{itemize}

The pair of numbers $(a, b)$ is considered the ciphertext.
\begin{example}[Encryption (Elgamal)]
Suppose we want to encrypt $M=6$. Choose $k = 7$. $a = 10^7 \mod 21 = 10$, $b = 13^7 \cdot 6 \mod 21 = 15$.
\label{ex:add:dm:elgamal_crypt}
\end{example}

\subsection{Decryption}
Knowing the private key $x$, we can restore the original message using
\begin{equation}
M = b\cdot \left(a^x\right)^{-1} \mod p,
\label{eq:add:dm:elgamal_decrypt}
\end{equation}
indeed because
\[
\left(a^x\right)^{-1} = g^{-kx} \mod p
\]
we have
\[
b\cdot \left(a^x\right)^{-1} = 
y^k M g^{-kx} = g^{kx} M g^{-kx} = M \mod p.
\]
\begin{example}[Decryption (Elgamal)]
The encrypted message from ex. \ref{ex:add:dm:elgamal_crypt} 
$C = (a=10, b=15)$. Using \eqref{eq:add:dm:elgamal_decrypt} we have
\[
M = 15 \cdot 13 \mod 21 = 6
\]
where it was used
\[
\left(10^3\right)^{-1} \equiv 13 \mod 21,
\]
since $13 \cdot 10^3 = 1 \mod 21$.
Thus, we restored the message $M=6$ encrypted in
ex. \ref{ex:add:dm:elgamal_crypt}.
\label{ex:add:dm:elgamal_crypt}
\end{example}