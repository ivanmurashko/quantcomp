\section{One-Time Pad}
\rindex{One-Time Pad}
The one-time pad scheme was proposed in 1917 by Major J. Maborn and G. Vernam. The classic one-time pad is a set of random keys, each of which is equal in size to the message being sent and is used only once.

Suppose we want to encrypt a message in a certain language (such as English). The number of characters (letters) used in the alphabet is denoted by $X$. For the English language (without punctuation and case distinction) $X = 26$. Then we assign a certain number $c$ to each character of the language, such that $0 \le c \le X$. For example, for the English language, we can write
\begin{equation}
\begin{array}{c}
A \rightarrow 0 \\
B \rightarrow 1 \\
\dots \\
Z \rightarrow 25 \\
\end{array}
\nonumber
\end{equation}

The encryption procedure \eqref{eqPart3CryptoEncryptClass} is described by the following expression
\begin{equation}
E_{K_i}\left(P_i\right) = P_i + K_i \mod X = C_i,
\label{eqPart3CryptoEncryptVernam}
\end{equation}
where $i$ is the number of the character being encrypted.

The decryption procedure \eqref{eqPart3CryptoDeEncryptClass} is described by the following expression
\begin{equation}
D_{K_i}\left(C_i\right) = C_i - K_i \mod X = P_i,
\label{eqPart3CryptoDeEncryptVernam}
\end{equation}
where $i$ is the number of the character being encrypted.

This procedure easily generalizes to the case of binary data, using the XOR operation ($a \oplus b$) instead of modulo addition for both encryption and decryption:
\begin{table}
\centering
\begin{tabular}{|c|c|c|}
\hline
$a$ & $b$ & $a \oplus b$ \\ \hline
0  & 0 & 0 \\
0  & 1 & 1 \\
1  & 0 & 1 \\
1  & 1 & 0 \\ \hline
\end{tabular}
\caption{XOR $a \oplus b$}
\label{tblXOR}
\end{table}

Claude Shannon showed \cite{bShenonCrypto} that if the key is truly random, has the same length as the original message, and is not reused, then the proposed one-time pad scheme is perfectly secure.

According to Shannon, perfect security can be defined as follows.
\begin{definition}
A cipher $\left(E, D\right)$ is perfectly secure if for any two messages of equal length $m_0$ and $m_1$, some ciphertext $c$, and key $k \leftarrow_R K$, the probabilities that the original text is $m_0$ or $m_1$ are equal:
\begin{equation}
P\left(E\left(m_0, k\right) = c \right) = 
P\left(E\left(m_1, k\right) = c \right)
\nonumber
\end{equation}
\end{definition}
Rephrasing this definition, it can be said that based on the original statistics of the ciphertext, no information about the original message can be obtained.

\begin{theorem}[Cryptographic Strength of the One-Time Pad]
The one-time pad scheme is perfectly secure.
\end{theorem}

\begin{proof}
Let $\left|K\right|$ denote the number of all possible keys of length $l$. Where $l$ is also the length of the original messages: $\left|m_{0,1}\right| = l$. Given that the key used to encrypt the message is uniquely defined:
\begin{equation}
k_{0,1} = c \oplus m_{0,1},
\nonumber
\end{equation} 
we obtain for the probabilities
\begin{equation}
P\left(E\left(m_0, k\right) = c \right) = 
P\left(E\left(m_1, k\right) = c \right) = 
\frac{1}{\left|K\right|}.
\nonumber
\end{equation}
\end{proof}

\section{Problems of Classical Cryptography}

If there is a perfectly secure cryptographic system (the one-time pad), then what is wrong with classical cryptography? The problem lies in obtaining keys that meet the requirements of the one-time pad (the key length equals the message length, the key consists of random data and is not reused) and transmitting these keys to Bob and Alice.

Problems arise both at the stage of key generation, \footnote{obtaining large sequences of random numbers is a non-trivial mathematical problem} and at the stage of transmitting these keys.

In classical cryptography, the so-called public key algorithms are used for key transmission. There are several key exchange protocols based on public key cryptographic systems. They are all based on the existence of two keys, the first of which, called the public key, is used only for encryption, and the second - the private key - for decryption. To obtain the private key from the public key, a complex mathematical operation must be performed. For example, the security of one of the most popular public key systems - RSA (see \ref{AddRSA}),
\rindex{RSA algorithm}
is based on the difficulty of factoring\footnote{decomposing into prime factors} large numbers.

The key distribution protocol scheme based on public key cryptography can be described as follows. In the first step, Alice creates public and private keys and sends the first one to Bob. Bob, in turn, creates the key that both Alice and Bob would like to have (which needs to be distributed). This key is encrypted (for example, using RSA) with Alice's public key and sent to her. Upon receiving this encrypted key, Alice can decrypt it using her private key.

If an eavesdropper (Eve) wants to learn the transmitted key, she must solve a complex mathematical problem of factoring large numbers. It is believed, but not proven, that the difficulty of factoring grows exponentially with the number of digits in the number \cite{bPhisQuantInfo}.\footnote{The fastest known algorithm solves the factoring problem of a number $N$ in time on the order of $O\left(exp\left(log^{\frac{1}{3}}N\left(log \, log N\right)^{\frac{2}{3}}\right)\right)$.} Thus, as the number of digits increases, the problem quickly becomes unsolvable.

There are several problems with this scheme. The first is that the complexity of factorization is not proven. Moreover, there are algorithms for quantum computers - Shor's algorithm (see \ref{Part4QuantCompShor}) - that solve the factoring problem for a number $N$ in time $O\left(log N\right)$, i.e., in time proportional to the number of digits in $N$. Therefore, at the moment when a quantum computer is built, all systems based on RSA\rindex{RSA algorithm}, will become obsolete.

%% Another problem makes key distribution systems based on public cryptography insecure. Imagine a situation where Eve can intercept all messages between Alice and Bob and replace them with her own, which is not entirely unsolvable in classical communication theory. In this case, Eve can impersonate Bob for Alice and Alice for Bob, thus fully controlling the message exchange between them and being able to decrypt the key during transmission.