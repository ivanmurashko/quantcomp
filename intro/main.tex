\chapter*{Introduction}
The purpose of this course is to introduce modern cryptography and how quantum mechanics can be used to solve complex cryptographic problems.

The course consists of 10 lectures, each lasting 1 hour.

The following topics are covered:
\begin{itemize}
\item Introduction to quantum mechanics (Lectures 1, 2)
\item Description of basic principles of quantum computing (Lecture 3)
\item Symmetric encryption algorithms and Grover's algorithm. (Lecture 4)
\item The classical RSA algorithm and its connection to the problem of finding the period of a function. (Lecture 5)
\item Discrete (classical) Fourier transform and its applications for finding the period of periodic functions. Implementation of the discrete Fourier transform on quantum elements is proposed (Lectures 6, 7)
\item Shor's algorithm for breaking RSA (Lecture 8)
\item Classical encryption algorithms based on the complexity of discrete logarithms. Modification of Shor's algorithm to solve the discrete logarithm problem (Lecture 9)
\item The final lecture, Lecture 10, is devoted to encryption algorithms based on elliptic curves. The ECDH algorithm is considered. Modification of Shor's algorithm for solving the discrete logarithm problem on elliptic curves is described.
\end{itemize}

Throughout the lectures, necessary mathematical explanations will be provided, such as
\begin{itemize}
\item Discrete mathematics: Fermat's little theorem, Euclidean algorithm, etc.
\item General algebra: concept of a group, Lagrange's theorem, cyclic group, concept of a field. Galois fields.
\item Linear algebra and matrix operations: matrix multiplication, linear operators, eigenvalues and eigenfunctions of linear operators
\item Classical probability theory: events, random variables, mean of a random variable
\end{itemize}

%% The purpose of this course is to introduce modern cryptography and how quantum mechanics can be used to solve complex cryptographic problems.

%% The first and second parts of the course provide an introduction to modern classical cryptography. The main focus is on asymmetric encryption methods.

%% The third part provides a brief introduction to quantum mechanics.

%% In the last, fourth part, the basics of quantum computing and their applications for solving classical cryptography problems are considered. The most well-known algorithms are described, such as Shor's algorithm, which allows for integer factorization in linear time, and Grover's algorithm, which performs a search in an unsorted array of data in time $O\left(\sqrt{N}\right)$.