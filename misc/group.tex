%% -*- coding:utf-8 -*-
\section{Introduction to Group Theory}
\label{sec:add:group}
\begin{definition}
\label{def:add:group}
A group $(\mathcal{G}, \circ)$ is a set of elements $g \in \mathcal{G}$ for which a certain binary operation $\circ$ is defined (often called multiplication or addition):
\begin{eqnarray}
\forall g_1,g_2 \in \mathcal{G},
\nonumber \\
g_1 \circ g_2 \in \mathcal{G}.
\label{addGroupMulDef}
\end{eqnarray}
The operation defined by \eqref{addGroupMulDef} has the property of associativity:
\begin{equation}
g_1 \circ \left( g_2 \circ g_3 \right ) = 
\left( g_1 \circ  g_2 \right ) \circ g_3.
\nonumber
\end{equation}
The set under consideration must contain an element $e_{\mathcal{G}}$ with the following property, valid for any element in the set $g$:
\begin{equation}
g \circ e_{\mathcal{G}} = e_{\mathcal{G}} \circ g = g.
\nonumber
\end{equation}
For each element of the group $g$, there must exist an inverse element $g^{-1} \in \mathcal{G}$ with the following property
\begin{equation}
g \circ g^{-1} = g^{-1} \circ g = e_{\mathcal{G}}
\nonumber
\end{equation} 
\end{definition}

\begin{definition}[Monoid]
\label{def:add:monoid}
If for a certain set of elements $\mathcal{G}$ the last property of a group (existence of an inverse element) is not satisfied, then this set is called a monoid or a semigroup.
\end{definition}

\begin{definition}[Abelian Group]
\label{def:add:abeliangroup}
A group $(\mathcal{A}, \circ)$ is called abelian or commutative if $\forall a_1,a_2 \in \mathcal{A}$: $a_1 \circ a_2 = a_2 \circ a_1$.
\end{definition}

\begin{example}
\emph{Group $\left(\mathbb{Z}, +\right)$}
The set of integers $\mathbb{Z} = \left\{0, \pm1, \pm2, \dots\right\}$ forms a group under the operation of addition.
\nonumber
\end{example}

\begin{definition}[Cyclic Group]
A cyclic group $G$ is a group generated by a single element $g: G = <g>$, i.e., all its elements are powers of $g$. The element $g$ is called the generator of the group $G$.
\label{def:add:algebra:cyclic_group}
\end{definition}

\begin{definition}[Multiplicative Group of the Ring of Residues]
Consider the set of integers coprime to $n$ and less than $n$, denoted by $\left(\mathbb{Z}/n\mathbb{Z}\right)^\times$. As the operation of multiplying two elements $a,b \in \left(\mathbb{Z}/n\mathbb{Z}\right)^\times$, we take
\[
a \circ b = a \cdot b \mod n.
\]
The unit element $e_{\left(\mathbb{Z}/n\mathbb{Z}\right)^\times}$ is $1$. Furthermore, it can be shown that for each $a \in \left(\mathbb{Z}/n\mathbb{Z}\right)^\times$ there exists $a^{-1} \in \left(\mathbb{Z}/n\mathbb{Z}\right)^\times$ such that $a \circ a^{-1} = 1$. Thus, $\left(\mathbb{Z}/n\mathbb{Z}\right)^\times$ is a group.
\label{def:add:algebra:mult_group}
\end{definition}

\begin{theorem}[On the Order of $\left(\mathbb{Z}/n\mathbb{Z}\right)^\times$]
The order of the group $\left(\mathbb{Z}/n\mathbb{Z}\right)^\times$ is determined by the following relation
\[
\left|\left(\mathbb{Z}/n\mathbb{Z}\right)^\times\right| = \phi(n),
\]
where $\phi(n)$ is \myref{def:add:discretmath:eulerfun}{Euler's function}. Moreover, if 
$n=p$ is a prime number, then
\[
\left|\left(\mathbb{Z}/p\mathbb{Z}\right)^\times\right| = \phi(p),
\]
and if $a \in \left(\mathbb{Z}/p\mathbb{Z}\right)^\times, a \ne 1$,
then $a$ is a generator of the group in question, i.e.,
\[
a^{p-1} = 1.
\]
\begin{proof}
TBD
\end{proof}
\label{thm:add:algebra:cyclic_mult_group}
\end{theorem}

\begin{theorem}[Lagrange]
\label{thm:lagrange}
For any finite group $\mathcal{G}$, the order (number of elements) of any subgroup $\mathcal{H}$ divides the order of $\mathcal{G}$:
\[
\left|\mathcal{G}\right| = h \left|\mathcal{H}\right|,
\] 
where the integer $h$ is called the index of the subgroup.
\end{theorem}