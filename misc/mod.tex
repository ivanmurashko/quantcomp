\section{Comparison by Modulus}
\begin{definition}
The notation
\begin{equation}
a \equiv b \mod{c}
\label{defAddMod}
\end{equation}
means that $a$ and $b$ have the same remainders when divided by $c$ or $a$ and $b$ are comparable by the modulus of the natural number $c$. Here, the number $c$ is called the modulus of comparison.
\end{definition}

Definition \ref{defAddMod} can also be interpreted as the difference $a - b$ being divisible by $c$.

\begin{example}
\emph{Comparison by Modulus}
$30 \equiv 8 \mod{11}$, because $30 = 2 \cdot 11 + 8$.
\end{example}

\begin{definition}[Negative Element]
If $a < n$, then $n - a$ will be called the element negative relative to $a$ and denoted by $-a \mod n$.
\end{definition}

\begin{example}
\emph{Negative Element}
\[
-5 \equiv 6 \mod 11,
\]
since $5 < 11, 6 = 11 - 5$.
\end{example}

\subsection{Arithmetic Operations}

\begin{lemma}[Addition by Modulus]
If $a_1 \equiv a_2 \mod n, b_1 \equiv b_2 \mod n$, then 
\[
a_1 + b_1 \equiv a_2 + b_2 \mod n
\]
\begin{proof}
We can write $a_1 = k_1 n + r_a, a_2 = k_2 n + r_a, b_1 = l_1 n + r_b, b_2 = l_2 n + r_b$ from which
\[
a_1 + b_1 = (k_1 + l_1) n + r_a + r_b \equiv r_a + r_b \mod n
\] 
and
\[
a_2 + b_2 = (k_2 + l_2) n + r_a + r_b \equiv r_a + r_b \mod n
\] 
from which
\[
a_1 + b_1 \equiv a_2 + b_2 \equiv r_a + r_b \mod n
\]
\end{proof}
\end{lemma}

\begin{lemma}[Multiplication by Modulus]
If $a_1 \equiv a_2 \mod n, b_1 \equiv b_2 \mod n$, then 
\[
a_1 \cdot b_1 \equiv a_2 \cdot b_2 \mod n
\]
\begin{proof}
If $a_1 \equiv a_2 \mod n, b_1 \equiv b_2 \mod n$, then 
\[
a_1 + b_1 \equiv a_2 + b_2 \mod n
\]
We can write $a_1 = k_1 n + r_a, a_2 = k_2 n + r_a, b_1 = l_1 n + r_b, b_2 = l_2 n + r_b$ from which
\[
a_1 \cdot b_1 = k_1 l_1 n + l_1 n r_a + k_1 n r_b + r_a r_b \equiv r_a r_b \mod n
\] 
and
\[
a_2 \cdot b_2 = k_2 l_2 n + l_2 n r_a + k_2 n r_b + r_a r_b \equiv r_a r_b \mod n
\] 
from which
\[
a_1 \cdot b_1 \equiv a_2 \cdot  b_2 \equiv r_a r_b \mod n
\]
\end{proof}
\end{lemma}

\subsection{Solving Equations}
\label{sec:add:discretmath:mod:equationsolve}
Very often in cryptography, one deals with equations of the form
\begin{equation}
a x \equiv b \mod n,
\label{eq:add:sicret:modeq}
\end{equation}
where $a, b, n$ are known integers, and $x$ is an unknown parameter to be determined.

It is obvious that if we find an integer $a^{-1}$, such that 
\[
a a^{-1} \equiv 1 \mod n,
\]
then
\[
x \equiv b a^{-1} \mod n.
\]

If $\gcd(a, n) = 1$, then in accordance with B\'ezout's identity (see \myref{thm:besu}{Bézout's theorem}) 
$\exists x, y: a x + n y = 1$, i.e., 
\[
x \equiv a^{-1} \mod n.
\]
Moreover, in accordance with remark \ref{rem:besu}, $a^{-1}$, and the solution to equation \eqref{eq:add:sicret:modeq}, can be found quite efficiently.

\subsection{Field $\mathbb{F}_p$}
\label{sec:add:diskretmath:mod:fp}
As we have seen in modular arithmetic, one can add, subtract, multiply, and even divide if compared by the modulus of a prime number. In this case, addition and multiplication operations are commutative and satisfy the distributive condition. Thus, the remainders form a field (see definition \ref{def:field}) which is called a Galois field and is denoted by $\mathbb{F}_p$.