\section{Fermat's Little Theorem}

\begin{theorem}[Fermat's Little Theorem]
\label{addDiscretSmallFerma}
If $p$ is a prime number, and $a$ is not divisible by $p$, then 
\begin{equation}
a^{p-1} \equiv 1 \mod{p}
\label{addDiscretSmallFermaEq}
\end{equation}
\end{theorem}

\begin{proof}
Consider the following relation
\[
a \cdot k_i \mod{p},
\]
where $k_i \subset \left\{1, \dots, p - 1\right\}$.

It is obvious that 
\begin{equation}
a \cdot k_i \equiv k_j \mod{p}.
\label{addDiscretSmallFermaInt}
\end{equation}
Indeed
\[
a \cdot k_i \mod{p} \subset \left\{1, \dots, p - 1\right\},
\]
as any remainder when divided by $p$ takes values $0, 1, \dots, p-1$. A zero remainder is impossible because $a$ and $p$ are coprime.

Additionally, each of the remainders $a \cdot k_i \mod{p}$ occurs only once. Assume that $a \cdot k_i \mod{p} = a \cdot k_j \mod{p}$ or $a \cdot \left( k_i - k_j \right ) \equiv 0 \mod{p}$, meaning $a$ is divisible by $p$, contradicting the coprimeness condition.

Multiplying all expressions from \autoref{addDiscretSmallFermaInt}, we get
\[
a \cdot 2a \cdot 3a \cdot \dots \cdot a\left(p - 1\right) 
\equiv 1 \cdot 2 \cdot 3 \cdot \dots \cdot \left(p - 1\right) \mod{p}.
\]
or
\[
a^{p-1} \left(p - 1\right)! \equiv \left(p - 1\right)! \mod{p},
\]
from which we obtain the required equality by the coprimeness of $p$ and $\left(p-1\right)!$:
\[
a^{p-1} \equiv 1 \mod{p}
\]
\end{proof}

\subsection{Pseudoprimes}
The generalization of Fermat's Little Theorem is not true, that is, if $a$ and $p$ are coprime numbers that satisfy the relation \autoref{addDiscretSmallFermaEq} then $p$ may not be prime. For example
\[
2 ^ {341 - 1} \equiv 1 \mod{ 341 },
\] 
even though $341 = 11 \cdot 31$.

Numbers $p$ that satisfy the relation \autoref{addDiscretSmallFermaEq}, but are not prime, are called pseudoprimes to the base $a$. For example, 341 is the first pseudoprime to the base 2.