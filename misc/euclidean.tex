%% -*- coding:utf-8 -*- 
\section{Greatest Common Divisor. Euclidean Algorithm}
\label{AddEuclidean}
\begin{definition}
The greatest common divisor of numbers $a$ and $b$ ($\mbox{GCD}\left(a, b\right)$) is the largest of their common divisors.
\label{defAddGCD}
\end{definition}

\begin{theorem}[Theorem on Common Divisors]
Suppose the following inequalities hold $a > b > 0$, and the number $r$ is the remainder of dividing $a$ by $b$. Thus, it can be written as
\begin{equation}
a = x \cdot b + r,
\label{eqAddGCD_a}
\end{equation}
where $x \ge 1$, $b > r \ge 0$. If $r=0$, then $b$ is the largest number that divides both $a$ and $b$ without a remainder. In case $r > 0$, then
\begin{equation}
\mbox{GCD}\left(a, b\right) = \mbox{GCD}\left(b, r\right).
\label{eqAddGCD_ab_br}
\end{equation}
\end{theorem}

\begin{proof}
To prove \eqref{eqAddGCD_ab_br} we show that any divisor of the pair $\left(a,b\right)$ is a divisor of the pair $\left(b,r\right)$. Let $d$ be a common divisor of the numbers $a$ and $b$, i.e., $a = d \cdot x_1$, $b = d \cdot x_2$. Thus, from \eqref{eqAddGCD_a} it follows
\begin{equation}
r = a - x \cdot b = d \cdot \left( x_1 - x \cdot x_2 \right),
\nonumber
\end{equation}
i.e., $d$ is a divisor of the number $r$.

Now we prove that any common divisor of the numbers $b$ and $r$ will be a common divisor of the numbers $a$ and $b$. Indeed, let $d$ be a common divisor of numbers $b$ and $r$, i.e., $b = y_1 \cdot d$ and $r = y_2 \cdot d$, thus \eqref{eqAddGCD_a} can be rewritten as
\begin{equation}
a = x \cdot y_1 \cdot d + y_2 \cdot d = d \cdot \left( x \cdot y_1 + y_2 \right),
\nonumber
\end{equation}
i.e., $d$ is a divisor of the number $a$.

Thus, the pairs of numbers $\left(a,b\right)$ and $\left(b,r\right)$ have common divisors, including the greatest divisor for one pair will be the same for the second.
\end{proof}

The relation \eqref{eqAddGCD_ab_br} leads to the following algorithm for computing the greatest common divisor
\begin{algorithm}
\caption{Euclidean Algorithm}
\begin{algorithmic}
    \STATE $a > b$
    \IF{$b = 0$} 
        \RETURN $a$
    \ENDIF
    \STATE $a \Leftarrow 0$
    \STATE $r \Leftarrow b$
    \STATE $b \Leftarrow a$
    \REPEAT
        \STATE $a \Leftarrow b$
        \STATE $b \Leftarrow r$
        \STATE $r \Leftarrow $ remainder of dividing $a$ by $b$
    \UNTIL{ ($r \ne 0$) }
    \RETURN $b$
\end{algorithmic}
\label{algAddEuclid}
\end{algorithm}

To estimate the complexity of algorithm \ref{algAddEuclid} we write it in the following form
\begin{eqnarray}
f_k = x_k \cdot f_{k + 1} + f_{k + 2},
\nonumber \\
f_0 = a, \: f_1 = b,
\nonumber \\
x_k \ge 1, \: f_k > f_{k+1} > f_{k + 2},
\nonumber
\end{eqnarray}
i.e., $f_k > 2 \cdot f_{k + 2}$, or $f_0 > 2 \cdot f_2 > \dots > 2^nf_{2n}$ meaning the algorithm will stop at $n = log_2\left(f_0\right) = log_2\left(a\right)$
\footnote{Further derivations assume that $log_2\left(a\right)$ is an integer}. The number of algorithm steps in this case is obviously $2n$ or $2 \cdot log_2\left(a\right)$. Thus, the algorithmic complexity of the Euclidean Algorithm can be written as $O\left(\log \left(a\right)\right)$.

\begin{example}
\emph{($\mbox{GCD}\left(2345,1456\right)$)}
%(%i5) gcd(2345,1456);
%(%o5)                                  7
%(%i6)      
%(%i8) mod(2345,1456);
%(%o8)                                 889
%(%i9) mod(1456,889);
%(%o9)                                 567
%(%i10) mod(889,567);
%(%o10)                                322
%(%i11) mod(567,322);
%(%o11)                                245
%(%i12) mod(322,245);
%(%o12)                                77
%(%i13) mod(245,77);
%(%o13)                                14
%(%i14) mod(77,14);
%(%o14)                                 7
%(%i15) mod(14,7);
%(%o15)                                 0
%(%i16) 
\begin{eqnarray}
2345 = 1456 + 889,
\nonumber \\
1456 = 889 + 567,
\nonumber \\
889 = 567 + 322,
\nonumber \\
567 = 322 + 245,
\nonumber \\
322 = 245 + 77,
\nonumber \\
245 = 3 \cdot 77 + 14,
\nonumber \\
77 = 5 \cdot 14 + 7,
\nonumber \\
14 = 2 \cdot 7.
\nonumber
\end{eqnarray}
Therefore, $\mbox{GCD}\left(2345,1456\right) = 7$. 
The number of algorithm steps is $8 < 2 \cdot log_2{2345} \approx 2 \cdot 11.2 = 22.4$.
\nonumber
\end{example}

\subsection{Bézout's Identity}

\begin{theorem}[Bézout]
\label{thm:besu}
If the numbers $a$ and $b$ are coprime, then the equation 
\[
ax + by = 1
\] 
has integer solutions.
\begin{proof}
We use the Euclidean Algorithm \ref{algAddEuclid} to find $\mbox{GCD}\left(a, b\right)$.
Assume $a > b$, then
\begin{eqnarray}
r_1 = a - b q_0, 
\nonumber \\
r_2 = b - r_1 q_1 = b - (a - b q_0)q_1 = b(1+q_1 q_0) - a q_1, 
\nonumber \\
r_3 = r_1 - r_2 q_2 = a - b q_0 - \left(b(1+q_1 q_0) - a q_1\right)q_2
= \nonumber \\
= a(1+q_1 q_2) - b(q_0 + q_2 + q_0 q_1 q_2),
\nonumber \\
\dots
\nonumber \\
\mbox{GCD}\left(a, b\right) = r_n = r_{n-2} - r_{n-1} q_{n-1} = \dots
= a x + b y, 
\label{eq:besu}
\end{eqnarray}
which proves our statement.
\end{proof}
\end{theorem}

\begin{remark}[On the Complexity of Computing Bézout's Identity]
\label{rem:besu}
The calculations in \eqref{eq:besu} are equivalent to the steps in algorithm \ref{algAddEuclid}. The number of these steps is $O\left(log_2(a)\right)$, thus the algorithm described by equations \eqref{eq:besu} is quite efficient and has complexity $O\left(log_2(a)\right)$.
\end{remark}

\begin{example}[Bézout's Identity]
\label{ex:besu}
Suppose $a = 25, b = 14$. The Euclidean algorithm formulas are
\begin{eqnarray}
11 = 25 - 14 \cdot 1,
\nonumber \\
3 = 14 - 11 \cdot 1,
\nonumber \\
2 = 11 - 3 \cdot 3,
\nonumber \\
1 = 3 - 2 \cdot 1.
\nonumber
\end{eqnarray}
From these formulas we get
\begin{eqnarray}
11 = 25 - 14 \cdot 1,
\nonumber \\
3 = 14 - 25 + 14 = 2 \cdot 14 - 25,
\nonumber \\
2 = 11 - 3 \cdot 3 = 25 - 14 - 3 \cdot (2 \cdot 14 - 25) = 4 \cdot 25 - 7 \cdot 14,
\nonumber \\
1 = 3 - 2 \cdot 1 = 9 \cdot 14 - 5 \cdot 25.
\nonumber
\end{eqnarray}
Thus 
\[
25 \cdot (-5)  + 14 \cdot 9 = 1.
\]
\end{example}