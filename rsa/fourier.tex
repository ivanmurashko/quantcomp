%% -*- coding:utf-8 -*- 
\section{Quantum Fourier Transform}
For the analysis of periodic sequences (functions), the discrete Fourier transform may be used (see \autoref{AddFourier}), which is defined by the following relation \eqref{eqAddFourierDiscretFourier}:
\begin{equation}
\tilde{X}_k = \sum^{M - 1}_{m = 0} x_m e^{-\frac{2 \pi}{M} k\cdot m},
\label{eqPart4QuantCompShorFourierDiscretFourier}
\end{equation}
where the original sequence of numbers $\left\{x_m\right\}$ has $M$ terms.

\subsection{Quantum Fourier Transform Scheme}
The quantum Fourier transform \footnote{The work \cite{DBLP:conf/new2an/2015} was used for analyzing the quantum Fourier transform scheme} deals with states of the form 
\begin{equation}
\ket{x} = \sum_{k = 0}^{M - 1}x_k \ket{k},
\label{eqPart4QuantCompShorQuantFourierSeries}
\end{equation}
where there is a sequence of amplitudes $\left\{x_k\right\}$ that defines the original sequence for the Fourier transform \eqref{eqPart4QuantCompShorFourierDiscretFourier}. The basis vector $\ket{k}$ records the number of the term in this sequence.

Obviously, the terms of the sequence \eqref{eqPart4QuantCompShorQuantFourierSeries} must meet the normalization condition
\[
\sum_k\left|x_k\right|^2 = 1.
\]

Assume that some operator $\hat{F}^{M}$ (quantum Fourier transform operator) transforms the basis vector $\ket{k}$ according to the rule defined by relation \eqref{eqAddFourierDiscretFourier}:
\begin{equation}
\hat{F}^{M}\ket{k} = \frac{1}{\sqrt{M}}\sum_{j = 0}^{M -1} e^{-i \omega k j}\ket{j}_{inv} 
\label{eqPart4QuantCompShorQuantFourierBasis}
\end{equation}
The systems of basis vectors $\left\{\ket{k}\right\}$ and $\left\{\ket{k}_{inv}\right\}$ represent the same set of vectors numbered differently.

From \eqref{eqPart4QuantCompShorQuantFourierSeries} and \eqref{eqPart4QuantCompShorQuantFourierBasis}, we obtain
\begin{eqnarray}
\hat{F}^{M}\ket{x} = \sum_{j = 0}^{M - 1}x_k \hat{F}^{M} \ket{k} = 
\nonumber \\
= \frac{1}{\sqrt{M}}\sum_{k = 0}^{M -1}\sum_{j = 0}^{M - 1} e^{-i \omega k j}x_k\ket{j}_{inv} = 
\nonumber \\
= \sum_{j = 0}^{M - 1} \left\{\frac{1}{\sqrt{M}}\left( \sum_{k = 0}^{M - 1}e^{-i \omega k j} x_k
\right)\right\}\ket{j}_{inv} = 
\nonumber \\
= \sum_{j = 0}^{M - 1}\tilde{X}_j\ket{j}_{inv} = \left|\tilde{X}\right>_{inv},
\nonumber
\end{eqnarray}
where 
\begin{equation}
\tilde{X}_j = \tilde{X}_j^{M} = 
\frac{1}{\sqrt{M}}\sum_{k = 0}^{M - 1}e^{-i \omega k j} x_k.
\label{eqPart4QuantCompShorQuantFourierEnd}
\end{equation}
Expression \eqref{eqPart4QuantCompShorQuantFourierEnd} repeats the classical analog \eqref{eqAddFourierDiscretFourier}, i.e., it can be written as:
\[
 \ket{x} \longleftrightarrow \left|\tilde{X}\right>_{inv}.
\]

\input ./rsa/figquantfourier0.tex

Now, let's assume that the input to our system is a state of the form \eqref{eqPart4QuantCompShorQuantFourierSeries} which represents a superposition of $M$ basis states $\left\{\ket{k}\right\}$ (see \autoref{figQuantCompQuantFourier0}). Let's assume that the number of basis states is a power of two, i.e., the basis state can be represented as the tensor product of $n = \log_2{M}$ qubits:
\begin{equation}
\ket{k} = \ket{a^{(k)}_0} \otimes  \ket{a^{(k)}_1} \otimes \cdots \otimes \ket{a^{(k)}_{n-1}}, 
\nonumber
\end{equation}
where
\begin{eqnarray}
k = a^{(k)}_0 + 2^1 a^{(k)}_1 + \dots + 2^{n-1} a^{(k)}_{n-1},
\nonumber \\
a^{(k)}_i \in \left\{0, 1\right\}.
\nonumber
\end{eqnarray}

At the output (see \autoref{figQuantCompQuantFourier0}), we have a superposition of $M$ basis states $\left\{\ket{j}_{inv}\right\}$, where the state $\ket{j}_{inv}$ is derived as:
\begin{equation}
\ket{j}_{inv} = \ket{b^{(j)}_{n-1}} \otimes \ket{b^{(j)}_{n-2}} \otimes \cdots \otimes \ket{b^{(j)}_{0}}, 
\nonumber
\end{equation}
where
\begin{eqnarray}
j = b^{(j)}_0 + 2^1 b^{(j)}_1 + \dots + 2^{n-1} b^{(j)}_{n-1},
\nonumber \\
b^{(j)}_i \in \left\{0, 1\right\}.
\nonumber
\end{eqnarray}

From formula \eqref{eqAddFourierDiscretFourierFFT}, it can be observed that if there is an input signal $x$ consisting of $n = \log_2{M}$ bits, then the bit $a^{(k)}_0$ can be used to select even (the first term of the sum \eqref{eqAddFourierDiscretFourierFFT}) or odd (the second term of the sum \eqref{eqAddFourierDiscretFourierFFT}) components.

In fact, excluding $a^{(k)}_0$, the state \eqref{eqPart4QuantCompShorQuantFourierSeries} can be represented as the sum of even and odd components: 
\begin{eqnarray}
\ket{x} = \sum_{k = 0}^{M - 1}x_k \ket{k} = 
\sum_{k = 0}^{M - 1}x_k \ket{a^{(k)}_0} \otimes  \ket{a^{(k)}_1} \otimes \cdots \otimes \ket{a^{(k)}_{n-1}} = 
\nonumber \\
 = \sum_{m = 0}^{\frac{M}{2} - 1}x_{k=2m} \ket{0} \otimes  \ket{a^{(k)}_1} \otimes \cdots \otimes \ket{a^{(k)}_{n-1}} +
\nonumber \\
+
\sum_{m = 0}^{\frac{M}{2} - 1}x_{k=2m + 1} \ket{1} \otimes  \ket{a^{(k)}_1} \otimes \cdots \otimes \ket{a^{(k)}_{n-1}} = 
\nonumber \\
 = \sum_{m = 0}^{\frac{M}{2} - 1}x_{2m} \ket{0} \otimes  \ket{m} +
\sum_{m = 0}^{\frac{M}{2} - 1}x_{2m + 1} \ket{1} \otimes  \ket{m} = 
\nonumber \\
= \sum_{m = 0}^{\frac{M}{2} - 1}x_{2m} \ket{2m} +
\sum_{m = 0}^{\frac{M}{2} - 1}x_{2m + 1} \ket{2m+1},
\nonumber
\end{eqnarray}
where
\begin{equation}
m = a^{(k)}_1 + 2^1 a^{(k)}_2 + \dots + 2^{n-2} a^{(k)}_{n-1}.
\nonumber
\end{equation}

\input ./rsa/figquantfourier1.tex

Applying the Fourier transform to only the higher bits $\hat{F}^{\frac{M}{2}}$, i.e., excluding $a^{(k)}_0$, we get (see \autoref{figQuantCompQuantFourier1}):
\begin{eqnarray}
\ket{x} \rightarrow
\hat{F}^{\frac{M}{2}} \sum_{m = 0}^{\frac{M}{2} - 1}x_{2m} \ket{2m} +
\hat{F}^{\frac{M}{2}} \sum_{m = 0}^{\frac{M}{2} - 1}x_{2m + 1}
\ket{2m+1} = 
\nonumber \\
=
\hat{F}^{\frac{M}{2}} \sum_{m = 0}^{\frac{M}{2} - 1}x_{2m} 
\ket{0} \otimes  \ket{m} +
\hat{F}^{\frac{M}{2}} \sum_{m = 0}^{\frac{M}{2} - 1}x_{2m + 1}
\ket{1} \otimes  \ket{m}
=
\nonumber \\
=
\sum_{m = 0}^{\frac{M}{2} - 1}x_{2m} 
\ket{0} \otimes \hat{F}^{\frac{M}{2}} \ket{m} +
\sum_{m = 0}^{\frac{M}{2} - 1}x_{2m + 1}
\ket{1} \otimes \hat{F}^{\frac{M}{2}} \ket{m}.
\label{eqPart4QuantCompShorQuantFourierStep1}
\end{eqnarray}
Considering expression \eqref{eqPart4QuantCompShorQuantFourierBasis}, we have
\begin{equation}
\hat{F}^{\frac{M}{2}} \ket{m} = \sqrt{\frac{2}{M}}
\sum_{j= 0}^{\frac{M}{2} - 1} e^{-i \frac{4 \pi}{M} m j}\ket{j}_{inv}.
\nonumber
\end{equation}
Thus, for \eqref{eqPart4QuantCompShorQuantFourierStep1}, we have
\begin{eqnarray}
\ket{x} \rightarrow
\sum_{m = 0}^{\frac{M}{2} - 1}x_{2m} 
\ket{0} \otimes \hat{F}^{\frac{M}{2}} \ket{m} +
\sum_{m = 0}^{\frac{M}{2} - 1}x_{2m + 1}
\ket{1} \otimes \hat{F}^{\frac{M}{2}} \ket{m} = 
\nonumber \\
=
\sqrt{\frac{2}{M}} \sum_{j = 0}^{\frac{M}{2} - 1} e^{-i \frac{4 \pi}{M} m j} 
\sum_{m = 0}^{\frac{M}{2} - 1}x_{2m} \ket{0} \otimes
\ket{j}_{inv}
+
\nonumber \\
+
\sqrt{\frac{2}{M}} \sum_{j = 0}^{\frac{M}{2} - 1} e^{-i \frac{4 \pi}{M} m j} 
\sum_{m = 0}^{\frac{M}{2} - 1}x_{2m+1} \ket{1} \otimes
\ket{j}_{inv}
=
\nonumber \\
=
\sum_{j = 0}^{\frac{M}{2} - 1}  
\left( \sqrt{\frac{2}{M}} 
\sum_{m = 0}^{\frac{M}{2} - 1} e^{-i \frac{4 \pi}{M} m j} x_{2m} 
\right) \ket{j}_{inv}
+
\nonumber \\
+
\sum_{j = 0}^{\frac{M}{2} - 1}
\left( \sqrt{\frac{2}{M}}  
\sum_{m = 0}^{\frac{M}{2} - 1}e^{-i \frac{4 \pi}{M} m j} x_{2m+1} 
\right)
\left|\frac{M}{2} + j\right>_{inv}
=
\nonumber \\
= \sum^{\frac{M}{2} - 1}_{j = 0}  \tilde{A}_{j} \ket{j}_{inv} +
\sum^{\frac{M}{2} - 1}_{j = 0}  \tilde{B}_{j} \left|\frac{M}{2} + j\right>_{inv},
\nonumber
\end{eqnarray}
where
\begin{eqnarray}
\tilde{A}_{j} = 
\sqrt{\frac{2}{M}} 
\sum_{m = 0}^{\frac{M}{2} - 1} e^{-i \frac{4 \pi}{M} m j} x_{2m} 
\nonumber \\
\tilde{B}_{j} =
\sqrt{\frac{2}{M}} 
\sum_{m = 0}^{\frac{M}{2} - 1} e^{-i \frac{4 \pi}{M} m j} x_{2m+1} 
\label{eqPart4QuantCompShorAB}
\end{eqnarray}

\input ./rsa/figquantfourier2.tex

If we now add a phase shift for odd elements, i.e., for those with $a_0^k = 1$, we get the scheme shown in \autoref{figQuantCompQuantFourier2}: 
\begin{eqnarray}
\ket{x} \rightarrow
\hat{F}^{\frac{M}{2}} \sum_{m = 0}^{\frac{M}{2} - 1}x_{2m} \ket{2m} +
\hat{R}\hat{F}^{\frac{M}{2}} \sum_{m = 0}^{\frac{M}{2} - 1} x_{2m + 1}
\ket{2m+1} =
\nonumber \\
= 
\sum^{\frac{M}{2} - 1}_{j = 0} \tilde{A}_{j} \ket{j}_{inv} +
\sum^{\frac{M}{2} - 1}_{j = 0}  
\tilde{B}_{j} \hat{R}\left|\frac{M}{2} + j\right>_{inv},
\nonumber \\
= 
\sum^{\frac{M}{2} - 1}_{j = 0}  \tilde{A}_{j} \ket{j}_{inv} +
\sum^{\frac{M}{2} - 1}_{j = 0}  
\tilde{C}_{j} \left|\frac{M}{2} + j\right>_{inv}.
\label{eqPart4QuantCompShorFourierStep2}
\end{eqnarray}
Using the expression
\begin{equation}
\hat{R}_l \ket{b^{(j)}_l} = 
exp{\left(- 2 \pi i \frac{b^{(j)}_l}{2^{n-l}}\right)}
\ket{b^{(j)}_l}
\nonumber
\end{equation}
we obtain that the operator $\hat{R}$ acts on the state $\left|\frac{M}{2} + j\right>_{inv}$ as follows:
\begin{eqnarray}
\hat{R}\left|\frac{M}{2} + j\right>_{inv} = 
\hat{R}\ket{1} \otimes  \ket{j}_{inv} = 
\nonumber \\
= 
\ket{1} \otimes \hat{R}_0 \ket{b^{(j)}_0}
\otimes \cdots \otimes \hat{R}_{n-2} \ket{b^{(j)}_{n-2}} = 
\nonumber \\
= 
\prod_{l = 0}^{n-2}exp{\left(- 2 \pi i \frac{2^lb^{(j)}_l}{2^n}\right)}
\ket{1} \otimes \ket{j}_{inv} = 
\nonumber \\
=
exp{\left(-2 \pi i \frac{j}{M}\right)}
\left|\frac{M}{2} + j\right>_{inv} 
\label{eqPart4QuantCompShorFourierStep2Pre}
\end{eqnarray}
In deriving \eqref{eqPart4QuantCompShorFourierStep2Pre}, it was taken into account that $j = b^{(j)}_0 + 2^1 b^{(j)}_1 + \dots + 2^{n-2} b^{(j)}_{n-2}$. 

Therefore, for $\tilde{C}_{j}$ in \eqref{eqPart4QuantCompShorFourierStep2}, we have:
\begin{eqnarray}
\tilde{C}_{j} = 
\sqrt{\frac{2}{M}} 
\sum_{m = 0}^{\frac{M}{2} - 1} 
e^{- 2 \pi i \frac{j}{M}}
e^{-i \frac{4 \pi}{M} m j} x_{2m+1} =
\nonumber \\
=
\sqrt{\frac{2}{M}} 
\sum_{m = 0}^{\frac{M}{2} - 1} 
e^{-i \frac{2 \pi}{M} \left(2m+1\right) j} x_{2m+1}
\label{eqPart4QuantCompShorC}
\end{eqnarray}

\input ./rsa/figquantfourier.tex

If we now apply the Hadamard transform \rindex{Hadamard Transform} to the qubit $\ket{a_0}$, we obtain the scheme shown in \autoref{figQuantCompQuantFourier}. The initial state then transforms according to the following law:
\begin{eqnarray}
\ket{x} \rightarrow
\hat{H}_0\hat{F}^{\frac{M}{2}} \sum_{m = 0}^{\frac{M}{2} - 1}x_{2m} \ket{2m} +
\hat{H}_0\hat{R}\hat{F}^{\frac{M}{2}}\sum_{m = 0}^{\frac{M}{2} - 1} x_{2m + 1} =
\nonumber \\
=
\sum_{j = 0}^{\frac{M}{2} - 1}
\tilde{A}_{j}
\hat{H}\ket{0} \otimes \ket{j}_{inv}
+
\sum_{j = 0}^{\frac{M}{2} - 1} 
\tilde{C}_{j}
\hat{H}\ket{1} \otimes \ket{j}_{inv} 
=
\nonumber \\
= 
\frac{1}{\sqrt{2}}\sum_{j = 0}^{\frac{M}{2} - 1}
\tilde{A}_{j} 
\left(\ket{0} + \ket{1} \right) \otimes  
\ket{j}_{inv}
+
\frac{1}{\sqrt{2}}\sum_{j = 0}^{\frac{M}{2} - 1}
\tilde{C}_{j} 
\left(\ket{0} - \ket{1} \right) \otimes  
\ket{j}_{inv}
=
\nonumber \\
=
\sum_{j = 0}^{\frac{M}{2} - 1}
\frac{\tilde{A}_{j} + \tilde{C}_{j} }{\sqrt{2}} 
\ket{0} \otimes \ket{j}_{inv} +
\sum_{j = 0}^{\frac{M}{2} - 1}
\frac{ \tilde{A}_{j} - \tilde{C}_{j}}{\sqrt{2}} 
\ket{1} \otimes \ket{j}_{inv}
=
\nonumber \\
=
\sum_{j = 0}^{\frac{M}{2} - 1}
\frac{\tilde{A}_{j} + \tilde{C}_{j} }{\sqrt{2}} \ket{j}_{inv} +
\sum_{j = 0}^{\frac{M}{2} - 1}
\frac{ \tilde{A}_{j} - \tilde{C}_{j}}{\sqrt{2}} 
\left|\frac{M}{2} + j \right>_{inv}.
\label{eqPart4QuantCompShorFourierStep3}
\end{eqnarray}
For the terms in \eqref{eqPart4QuantCompShorFourierStep3}, considering the equalities \eqref{eqPart4QuantCompShorAB} and \eqref{eqPart4QuantCompShorC}, we have:
\begin{eqnarray}
\frac{\tilde{A}_{j} + \tilde{C}_{j} }{\sqrt{2}} = 
\sqrt{\frac{1}{M}} 
\sum_{m = 0}^{\frac{M}{2} - 1} e^{-i \frac{4 \pi}{M} m j} x_{2m}  +
\sqrt{\frac{1}{M}} 
\sum_{m = 0}^{\frac{M}{2} - 1} 
e^{-i \frac{2 \pi}{M} \left(2m+1\right) j} x_{2m+1} = 
\nonumber \\
=
\sqrt{\frac{1}{M}} \sum_{m = 0}^{M - 1}
e^{-i \frac{2 \pi}{M} m j} x_{m}
\label{eqPart4QuantCompShorFourierStep3_1}
\end{eqnarray}
and
\begin{eqnarray}
\frac{\tilde{A}_{j} - \tilde{C}_{j} }{\sqrt{2}} = 
\sqrt{\frac{1}{M}} 
\sum_{m = 0}^{\frac{M}{2} - 1} e^{-i \frac{4 \pi}{M} m j} x_{2m}  -
\sqrt{\frac{1}{M}} 
\sum_{m = 0}^{\frac{M}{2} - 1} 
e^{-i \frac{2 \pi}{M} \left(2m+1\right) j} x_{2m+1}
= 
\nonumber \\
=
\sqrt{\frac{1}{M}} \sum_{m = 0}^{M - 1}
e^{-i \frac{2 \pi}{M} m j} x_{m} \frac{1 + e^{-i \pi m}}{2}
-
\sqrt{\frac{1}{M}} \sum_{m = 0}^{M - 1}
e^{-i \frac{2 \pi}{M} m j} x_{m} \frac{1 - e^{-i \pi m}}{2} 
=
\nonumber \\
=
\sqrt{\frac{1}{M}} \sum_{m = 0}^{M - 1}
e^{-i \frac{2 \pi}{M} m j} e^{-i \pi m } x_{m} 
=
\sqrt{\frac{1}{M}} \sum_{m = 0}^{M - 1}
e^{-i \frac{2 \pi}{M} m j} e^{-i \frac{2 \pi}{M} m \frac{M}{2} } x_{m} 
=
\nonumber \\
=
\sqrt{\frac{1}{M}} \sum_{m = 0}^{M - 1}
e^{-i \frac{2 \pi}{M} m \left(\frac{M}{2} + j\right)} x_{m}
\label{eqPart4QuantCompShorFourierStep3_2}
\end{eqnarray}

Combining \eqref{eqPart4QuantCompShorFourierStep3}, \eqref{eqPart4QuantCompShorFourierStep3_1}, and \eqref{eqPart4QuantCompShorFourierStep3_2}, we finally obtain 
\begin{eqnarray}
\ket{x} \rightarrow
\sum_{j = 0}^{\frac{M}{2} - 1} \sqrt{\frac{1}{M}} \sum_{m = 0}^{M - 1}
e^{-i \frac{2 \pi}{M} m j} x_{m} \ket{j}_{inv} +
\nonumber \\
+
\sum_{j = 0}^{\frac{M}{2} - 1} \sqrt{\frac{1}{M}} \sum_{m = 0}^{M - 1}
e^{-i \frac{2 \pi}{M} m \left(\frac{M}{2} + j\right)} x_{m} 
\left|\frac{M}{2} + j\right>_{inv} =
\nonumber \\
= \sum_{j = 0}^{M - 1} \tilde{X}_j^{M} \ket{j}_{inv}
\nonumber
\end{eqnarray}
