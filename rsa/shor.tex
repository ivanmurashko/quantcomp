%% -*- coding:utf-8 -*- 
\section{Shor's Algorithm}
\label{Part4QuantCompShor}
One of the most popular RSA encryption algorithms (see 
\autoref{AddRSA}) \rindex{RSA algorithm}
is based on the assumption of the complexity of factorization
(factoring into prime numbers) of large numbers. Therefore,
algorithms that allow performing factorization into prime numbers
are of particular interest. Below is the description of such
an algorithm proposed by Shor \cite{bShor94}.

\subsection{Number Factorization and Period Finding for Functions}
\label{sec:rsa:algoshor:periodfind}
The problem of factorizing a certain number $N$ is closely related to finding the period
of functions. Consider the following, which is called the modular exponentiation function
\begin{equation}
f\left(x, a\right) = a^x \mod N.
\label{eqRsaQuantCompShorClassPart}
\end{equation}
The function \eqref{eqRsaQuantCompShorClassPart} depends on
the analyzed number $N$ and two arguments $x$ and $a$. The argument $a$
is chosen from the following conditions
\begin{eqnarray}
0 < a < N,
\nonumber \\
\gcd\left(N, a\right) = 1.
\label{eqRsaQuantCompShorAConditions}
\end{eqnarray}
A typical graph of the function \eqref{eqRsaQuantCompShorClassPart} is presented in
\autoref{picRsaQuantCompShorClassPart}.

% maxima a = 2 N = 21
%draw2d(yrange=[0, 18], points_joined = true, point_type = 6,
%point_size = 2, points(makelist([x, power_mod(2, x, 21)], x, 0, 30)),
%user_preamble="set output 'picshorclass.tex'; set terminal latex; set
%xlabel '$x$'; set ylabel '$f(x)$'");

\input ./rsa/figshorclass.tex

The conditions for selecting the coefficient $a$
\eqref{eqRsaQuantCompShorAConditions} are such that $a$ and $N$ do not have
common divisors. If such divisors exist, they are the desired solution to the factorization problem and can be easily found using the Euclidean algorithm (see \autoref{AddEuclidean}).

The function \eqref{eqRsaQuantCompShorClassPart} is periodic,
i.e., there exists a number $r$ such that $f\left(x + r, a\right) = 
f\left(x, a\right)$. The smallest possible number $r$ is called
the period of the function \eqref{eqRsaQuantCompShorClassPart}. 

To prove the periodicity, note that $f\left(x, a\right)$ cannot
be equal to zero. Indeed, if the condition
$f\left(x, a\right) = 0$ is met, then 
\[
\exists x \in \left\{0, 1, \dots\right\}:
a^x = k \cdot N,
\]
where $k$ is an integer, which is not possible due to
the coprimeness of $a$ and $N$ \eqref{eqRsaQuantCompShorAConditions}
\footnote{Here it is obviously assumed that $N > 1$}.

Thus, the range of values of function
\eqref{eqRsaQuantCompShorClassPart} is limited to the set 
\begin{equation}
f\left(x,
a\right) \in \left\{1, \dots, N - 1\right\},
\nonumber
\end{equation}
hence 
\begin{equation}
\exists k,j: k > j, k,j \in \left\{0, 1, \dots, N\right\},
f\left(k,a\right) = f\left(j,a\right),
\nonumber
\end{equation}
which proves the periodicity of the function \eqref{eqRsaQuantCompShorClassPart}.

Let $k = j + r$, then
\[
a^k \mod{N} = a^{j + r} \mod{N} = a^j a^r \mod{N}= a^j \mod{N},
\]
since $a$ and $N$ are coprime, we can write
\begin{equation}
a^r \equiv 1 \mod{N}.
\label{eqRsaQuantCompShorPeriodConditions}
\end{equation}


The period of the function \eqref{eqRsaQuantCompShorClassPart} can be either
even or odd. In Shor's algorithm, we are interested in the former case:
the period is an even number. Otherwise, a new number $a$ is chosen and the period finding is repeated. Thus, considering $r= 2\cdot l$ we
can rewrite \eqref{eqRsaQuantCompShorPeriodConditions} as
\begin{equation}
a^{2 \cdot l} \equiv 1 \mod{N},
\nonumber
\end{equation}
and since $r$ is the smallest number satisfying the periodicity condition, then
\[
a^{l} \not\equiv 1 \mod{N}.
\]
If, at the same time, a number $a$ is chosen such that 
\[
a^{l} \not\equiv -1 \mod{N},
\]
then we have 
\begin{equation}
\left(a^l - 1\right)\left(a^l + 1\right) = k \cdot N,
\label{eqRsaQuantCompShorPeriodFinal}
\end{equation}
where $k$ is some integer. From
\eqref{eqRsaQuantCompShorPeriodFinal} we find that $a^l \pm 1$ have
common non-trivial (different from 1) divisors with $N$.

\begin{example}
\emph{Finding the Divisors of the Number $N=21$}
\label{exRsaQuantCompShorGCD}
As an example, consider the problem of finding the divisors of $N =
21$. Choosing $a=2$, we obtain the period of the function
\eqref{eqRsaQuantCompShorClassPart} $r = 6$ (see
\autoref{picRsaQuantCompShorClassPart}). 
Obviously,
\[
2^3 \equiv 8 \mod{21} \not\equiv -1 \mod{21}.
\]
Thus, by finding the corresponding greatest common divisors, we solve
the problem
\begin{eqnarray}
\gcd\left( 2^3 - 1, 21 \right) = \gcd\left( 7, 21 \right)
= 7,
\nonumber \\
\gcd\left( 2^3 + 1, 21 \right) = \gcd\left( 9, 21 \right)
= 3,
\nonumber \\
21 = 7 \cdot 3.
\nonumber
\end{eqnarray}
\end{example}

Thus, the problem of factorizing the number $N$ can be reduced to
the problem of finding the period of a certain function using the following
algorithm:
\rindex{Shor's algorithm}
\input rsa/shoralgo.tex
