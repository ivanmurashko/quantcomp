\section{Discrete Fourier Transform}
\label{AddFourier}

The Fourier transform plays an important role in digital signal processing, particularly for analyzing periodic sequences.
\subsection{Definition}
\begin{definition}
Assume there are $M$ samples $x_0, x_1, \dots, x_{M-1}$, then the discrete Fourier transform is given by the following relation
\begin{equation}
\tilde{X}_k = \frac{1}{\sqrt{M}}\sum^{M - 1}_{m = 0} x_m e^{-\frac{2 \pi i}{M} k\cdot m},
\label{eqAddFourierDiscretFourier}
\end{equation}
which is also written as
\begin{equation}
\left\{x_m\right\} \longleftrightarrow \left\{\tilde{X}_k\right\}.
\nonumber
\end{equation}
The inverse Fourier transform can be obtained using a similar relation
\begin{equation}
x_k = \frac{1}{\sqrt{M}}\sum^{M - 1}_{m = 0} \tilde{X}_m e^{\frac{2 \pi i}{M} k\cdot m},
\nonumber
\end{equation}
\end{definition}

In \autoref{picAddFourierFourier}, a graph of a certain periodic function and its Fourier transform is shown. 

\input ./rsa/figfourier.tex

Expression \eqref{eqAddFourierDiscretFourier} can also be rewritten in matrix form
\begin{equation}
\vec{\tilde{X}} = \hat{F} \vec{x},
\nonumber
\end{equation}
where
\begin{equation}
\vec{x} = 
\left(
\begin{array}{c}
x_0 \\
x_1 \\
\vdots \\
x_{M-1}
\end{array}
\right)
,
\vec{\tilde{X}} = 
\left(
\begin{array}{c}
\tilde{X}_0 \\
\tilde{X}_1 \\
\vdots \\
\tilde{X}_{M-1}
\end{array}
\right)
,
\nonumber
\end{equation}
and the matrix $\hat{F}$ is given by
\begin{equation}
\hat{F} = 
\frac{1}{\sqrt{M}}
\begin{pmatrix}
1 & 1 & 1 & \cdots & 1 \\
1 & e^{-i \omega} & e^{-2 i \omega} & \cdots & 
e^{-\left( M - 1 \right) i \omega} \\
1 & e^{-2 i \omega} & e^{-4 i \omega} & \cdots & 
e^{-2 \left( M - 1 \right) i \omega} \\
1 & e^{-3 i \omega} & e^{-6 i \omega} & \cdots & 
e^{-3 \left( M - 1 \right) i \omega} \\
\vdots & \vdots & \vdots & \ddots & \vdots \\
1 & e^{-\left( M - 1 \right) i \omega} & e^{-2\left( M - 1 \right) i \omega} & \cdots & 
e^{- \left( M - 1 \right)\left( M - 1 \right) i \omega} \\
\end{pmatrix}
,
\label{eqAddFourierDiscretFourierMatrixElem}
\end{equation}
where
\[
\omega = \frac{2 \pi}{M}.
\]
For the matrix element of the matrix
\eqref{eqAddFourierDiscretFourierMatrixElem}, we can write
\begin{equation}
F_{n m} = \frac{1}{\sqrt{M}}e^{-i \omega n m},
\label{eqAddFourierDiscretFourierMatrixElem2}
\end{equation}
where $n, m \in \{ 0, 1, \dots, M - 1\}$.
