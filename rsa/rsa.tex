%% -*- coding:utf-8 -*- 
\section{RSA Algorithm}
\label{AddRSA}
\rindex{RSA algorithm}
The RSA algorithm (abbreviation from the last names Rivest, Shamir, and Adleman) is 
an asymmetric encryption algorithm
\footnote{An asymmetric (public key) encryption algorithm uses two different keys: one for encryption and the other for decryption},  
based on the complexity of factorizing a number into prime factors.  

\subsection{Key Generation}
\rindex{RSA algorithm!key generation}
Consists of several steps
\input ./rsa/rsagenalgo.tex

The original numbers $p$ and $q$ are kept secret, as with their help,
the calculation of $\phi(n)$ becomes trivial.

It should be noted that to obtain the private key from the public key,
one must calculate $\phi(n)$ given $n$. This task is difficult (if $p$ and $q$ are unknown), as noted in comment \ref{rem:add:discretmath:eulerfuncomplex}. 

\input ./rsa/rsagenex.tex

\subsection{Encryption}
\rindex{RSA algorithm!encryption}
\input ./rsa/rsaencrypt.tex

\subsection{Decryption}
\rindex{RSA algorithm!decryption}
\input ./rsa/rsadecrypt.tex

\subsection{Proof}
We want to prove that 
\[
\left(m^e\right)^d \equiv m \mod{p \cdot q}
\]
for any positive number $m$ when $p$ and $q$ are prime numbers, and $e$
and $d$ satisfy the expression
\[
d \cdot e \equiv 1 \mod{\phi\left(p \cdot q\right)},
\]
which we can rewrite as
\[
d \cdot e - 1 = h \left(p - 1\right)\left(q - 1\right).
\]

Thus,
\[
m^{e\cdot d} =m m^{h \left(p - 1\right)\left(q - 1\right)}.
\]
Then there are two cases: when $m$ is divisible by $p$ and when $m$ and $p$
are coprime.

In the first case 
\[
m^{e\cdot d} \equiv m \equiv 0 \mod{p}
\]
In the second case, we use
\myref{addDiscretSmallFerma}{Fermat's Little Theorem}:
\[
m m^{h \left(p - 1\right)\left(q - 1\right)} 
= m \left(m^{p - 1}\right)^{h \left(q - 1\right)} \equiv m \cdot 1^{h
  \left(q - 1\right)} \equiv m \mod{p}.
\]
Similarly, we have either
\[
m^{e\cdot d} \equiv m \equiv 0 \mod{q}
\]
or due to Fermat's Little Theorem
\[
m m^{h \left(p - 1\right)\left(q - 1\right)} 
= m \left(m^{q - 1}\right)^{h \left(p - 1\right)} \equiv m \cdot 1^{h
  \left(p - 1\right)} \equiv m \mod{q}.
\]
Thus, we have the following two types of relationships:
the trivial equality
\begin{eqnarray}
x_1 = m \equiv m \mod p,
\nonumber \\
x_1 = m \equiv m \mod q,
\nonumber
\end{eqnarray}
and the newly obtained relationships
\begin{eqnarray}
x_2 = m^{ed} \equiv m \mod p,
\nonumber \\
x_2 = m^{ed} \equiv m \mod q,
\nonumber
\end{eqnarray}
from which, by virtue of \myref{thm:chineseremainder}{the Chinese Remainder Theorem}
\[
x_1 \equiv x_2 \mod p \cdot q
\]
i.e.
\[
m^{e\cdot d} \equiv m \mod n
\]
